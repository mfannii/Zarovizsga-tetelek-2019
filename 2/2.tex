\documentclass[12pt]{article}
\setlength{\textwidth}{17cm}
\setlength{\textheight}{24cm}
\setlength{\topmargin}{-2cm}
\setlength{\footskip}{1cm}
\setlength{\evensidemargin}{0cm}
\setlength{\oddsidemargin}{0cm}
\setlength{\parindent}{0cm}

\usepackage{allrunes}
\usepackage{amsmath}
\usepackage[magyar]{babel}
\usepackage[T1]{fontenc}
\usepackage[utf8]{inputenc}
\usepackage{fixltx2e}
\usepackage{multirow}

\usepackage[hyphens]{url}
\usepackage[unicode,colorlinks=true,breaklinks]{hyperref}
%\usepackage[dvips]{hyperref}
%should display links, but it does not work with \H accent
%and formulas in section titles

\hypersetup{colorlinks,linkcolor=blue,urlcolor=magenta,citecolor=magenta}
%Breaks long url`s in text, while keeping it one link:

\usepackage{amsfonts}
\usepackage{amsthm}
\usepackage{amssymb}


\theoremstyle{plain}
\usepackage{graphicx}

%\usepackage{gensymb}
\usepackage{float}

% For bra-ket notation
\usepackage{braket}

%% New commands
\newcommand{\dd}{\textrm{d}}

%% Pauli matrices
\newcommand{\sigx}{\sigma_x}
\newcommand{\sigy}{\sigma_y}
\newcommand{\sigz}{\sigma_z}

\newcommand{\paulix}{
    \left( \begin{array}{cc}
        0 & 1 \\
        1 & 0
    \end{array}
    \right)
}

\newcommand{\pauliy}{
    \left( \begin{array}{cc}
        0 & -i \\
        i & 0
    \end{array}
    \right)
}

\newcommand{\pauliz}{
    \left( \begin{array}{cc}
        1 & 0 \\
        0 & -1
    \end{array}
    \right)
}


\begin{document}
\title{2. tétel}
\author{Nagy Dániel}
\maketitle


\newpage
\begin{abstract}
    Bootstrap módszerek. A maximum likelihood módszer. Hipotézis tesztelés. Extrém statisztikák.
    Post hoc analízis. Regresszió. Függetlenségvizsgálat. Egzakt tesztek.
\end{abstract}

\section{Alapfogalmak}
\begin{itemize}
    \item Eseménytér (ez egy abstrakt fogalom): $\Omega = \{\omega_1, \omega_2, ..., \omega_n\}$ pl. kockadobás esetén $\Omega = \{ \omega_1=\text{"1est dobok"}, \omega_2=\text{"2est dobok"}, \omega_3=\text{"párosat dobok"} ... \}$
    \item Valószínűségi változó: $X:\Omega \rightarrow \mathbb R$ pl. kockadobás esetén $X(\omega_1) = 1, X(\omega_2) = 2, ... $
    \item Valószínűség: $P$ egy mérték, amely $\Omega$ részhalmazaihoz számot rendel:
        \begin{itemize}
            \item $P: \mathcal{P}(\Omega) \rightarrow \mathbb R$
            \item $P(\Omega) = 1$ és $P(\varnothing) = 0$
            \item $ 0 \leq P(A) \leq 1 ~ \forall A \in \Omega$
            \item Ha $A_1, A_2, ...$ diszjunkt részhalmazai $\Omega$-nak, akkor 
            \begin{equation*}
                P\left(\bigcup\limits_{i=1}^{\infty}A_i\right) = \sum\limits_{i=1}^{\infty}P(A_i)
            \end{equation*}
        \end{itemize}
    \item Hasznos összefüggések:
        \begin{itemize}
            \item $P(A\cup B) = P(A)+P(B)-P(A\cap B)$
            \item Két esemény független $\Longleftrightarrow P(A\cap B) = P(A)P(B)$
            \item $P(A|B) = \frac{P(A\cap B)}{P(B)}$
            \item Teljes valószínűség: Ha $A_1, A_2, ...$ az $\Omega$ egy felosztása, akkor 
                \begin{equation*}
                    P(B) = \sum\limits_{k} P(B|A_k)P(A_k)
                \end{equation*}
            \item Bayes-tétel: Ha $A_1, A_2, ...$ az $\Omega$ egy felosztása, akkor 
                \begin{equation*}
                    P(A_k|B) = \frac{P(B|A_k)P(A_k)}{P(B)} = \frac{P(B|A_k)P(A_k)}{\sum\limits_{j} P(B|A_j)P(A_j)}
                \end{equation*}
        \end{itemize}
    \item Eloszlásfüggvény (CDF - cumulative distribution function):
        \begin{equation*}
            F_X(x) = P(X<x) = P(\{\omega\in\Omega | X(\omega)<x \})
        \end{equation*}
        diszkrét esetben 
        \begin{equation*}
            F_X(x) = P(X=x) = P(\{\omega\in\Omega | X(\omega)=x \})
        \end{equation*}
    \item Sűrűségfüggvény (PDF - Probability density function):\\
    Ha az $X$ változó eloszlásfüggvénye $F_X(x)$, akkor a sűrűségfüggvény definíciója
        \begin{equation*}
            F_X(x) = \int\limits_{-\infty}^{x}\rho_X(\xi)\dd \xi \Longleftrightarrow P(a \leq X(\omega) \leq b) = \int\limits_{a}^{b}\rho_X(x)\dd x
        \end{equation*}
        Megjegyzés: sűrűségfüggvénye csak folytonos eloszlású valószínűségi változónak van.
    \item Várható érték
    \begin{equation*}
        \text{folytonos eset}~E(X) = \langle X \rangle = \int\limits_{-\infty}^{\infty}x\rho(x) \dd x 
    \end{equation*}
    \begin{equation*}
        \text{diszkrét eset}~E(X) = \langle X \rangle = \sum\limits_{k} x_k P(X=x_k)
    \end{equation*}
    \item Várható értékre vonatkozó azonosságok:
        \begin{itemize}
            \item Ha $Y=g(X) \Rightarrow E(Y) = E(g(X)) = {\displaystyle\int\limits_{-\infty}^{\infty}g(x)\rho(x) \dd x}$  
            \item $E\left(\sum\limits_k a_k X_k\right) = \sum\limits_k a_k E(X_k)$
            \item Ha $X_1, X_2, ...$ független változók, akkor ${\displaystyle E\left(\prod\limits_k X_k\right) = \prod\limits_k E(X_k)}$
        \end{itemize}  
\end{itemize}

\section{Bootstrap módszerek}
\subsection{Jackknife módszer}
\section{Maximum likelihood}
\section{Extrém statisztikák}
\section{Post-hoc analízis}
\section{Regresszió}
\section{Hipotézis tesztelés}
\section{z-teszt, t-test}
\subsection{Konfidenciaintervallumok}
\subsection{Függetlenségvizsgálat, $\chi^2$-próba}

\bibliographystyle{plain}
\bibliography{references}

\end{document}
