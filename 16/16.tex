\documentclass[12pt]{article}
\setlength{\textwidth}{17cm}
\setlength{\textheight}{24cm}
\setlength{\topmargin}{-2cm}
\setlength{\footskip}{1cm}
\setlength{\evensidemargin}{0cm}
\setlength{\oddsidemargin}{0cm}
\setlength{\parindent}{0cm}

\usepackage{allrunes}
\usepackage{amsmath}
\usepackage[magyar]{babel}
\usepackage[T1]{fontenc}
\usepackage[utf8]{inputenc}
\usepackage{fixltx2e}
\usepackage{multirow}

\usepackage[hyphens]{url}
\usepackage[unicode,colorlinks=true,breaklinks]{hyperref}
%\usepackage[dvips]{hyperref}
%should display links, but it does not work with \H accent
%and formulas in section titles

\hypersetup{colorlinks,linkcolor=blue,urlcolor=magenta,citecolor=magenta}
%Breaks long url`s in text, while keeping it one link:

\usepackage{amsfonts}
\usepackage{amsthm}
\usepackage{amssymb}


\theoremstyle{plain}
\usepackage{graphicx}

%\usepackage{gensymb}
\usepackage{float}

% For bra-ket notation
\usepackage{braket}

%% New commands
\newcommand{\dd}{\textrm{d}}

%% Pauli matrices
\newcommand{\sigx}{\sigma_x}
\newcommand{\sigy}{\sigma_y}
\newcommand{\sigz}{\sigma_z}

\newcommand{\paulix}{
    \left( \begin{array}{cc}
        0 & 1 \\
        1 & 0
    \end{array}
    \right)
}

\newcommand{\pauliy}{
    \left( \begin{array}{cc}
        0 & -i \\
        i & 0
    \end{array}
    \right)
}

\newcommand{\pauliz}{
    \left( \begin{array}{cc}
        1 & 0 \\
        0 & -1
    \end{array}
    \right)
}


\begin{document}
\title{16. tétel}
\author{valaki}

\maketitle


\newpage
\begin{abstract}
    Többdimenziós adatok – Geográfiai és térbeli adatok reprezentálása, pontfelhők. Keresési alapproblémák: intervallum-keresés, térbeli keresés, legközelebbi szomszédok. Térbeli indexek, térkitöltő görbék (Z-index, Peano–Hilbert-index), kD-fa, R-fa, a bitkódolás szerepe. Tércellázási módszerek: Delaunay-háromszögelés, Voronoi-cellázás. A gömb indexelése, Quad-tree, HEALPix, HTM.
    
    Multidimensional data - geographical and spatial data representation, point clouds. Basic searching algorithms: interval searching, spatial searching, nearest neighbours. Spatial indices, space filling curves (Z-index, Peano–Hilbert-index), kD-tree, R-tree, the role of binary coding. Spatial tesselation: Delaunay-triangulation, Voronoi-diagram. Pixelization of the sphere, quad-tree, HEALPix, HTM.
\end{abstract}


\section{Spatial data representation}
https://gisgeography.com/spatial-data-types-vector-raster/

+egy előadásból, letöltve


\section{Searching Algorithms}

%https://www.geeksforgeeks.org/searching-algorithms/

+nearest neighbours
%https://en.wikipedia.org/wiki/Nearest_neighbor_search

\section{Spatial indices}
%https://en.wikipedia.org/wiki/Spatial_database

\section{Space filling curves}
%https://en.wikipedia.org/wiki/Space-filling_curve

%http://wwwmayr.informatik.tu-muenchen.de/konferenzen/Jass05/courses/2/Valgaerts/Valgaerts_paper.pdf

%https://www.forceflow.be/2013/10/07/morton-encodingdecoding-through-bit-interleaving-implementations/

%https://en.wikipedia.org/wiki/Binary_code

\section{Spatial tesselation}

%https://www.geos.ed.ac.uk/~gisteac/gis_book_abridged/files/ch36.pdf

%http://aseldawy.blogspot.com/2015/12/voronoi-diagram-and-dealunay.html

\subsection{Delaunay, Voronoi}
%https://en.wikipedia.org/wiki/Delaunay_triangulation

%https://en.wikipedia.org/wiki/Voronoi_diagram


\subsection{Pixelization of the sphere}

%https://rts2.org/malaga/pres/L_Nicastro.pdf


\newpage

\begin{thebibliography}{Nature}
%\bibliography{references}

\bibitem{Deeplea_Goodfellow}\hypertarget{Deeplea_Goodfellow}{}
Ian Goodfellow and Yoshua Bengio and Aaron Courville (2016). \textit{Deep learning}. MIT Press. \url{http://www.deeplearningbook.org}

\end{thebibliography}

\end{document}
